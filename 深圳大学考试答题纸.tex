\documentclass[a4paper,zihao=5,UTF8]{ctexart}
\usepackage[margin=1in]{geometry}
\usepackage{fancyhdr}
\usepackage{amsmath}
\usepackage{amssymb}
\usepackage{comment}

\usepackage[hidelinks]{hyperref} %生成引用链接

% 颜色字体
\usepackage{color}

% 引用(编译需要使用xe->bib->xe->xe)
\usepackage{cite}

\usepackage{lastpage}

\pagestyle{fancy}
\lhead{\author}
\chead{\date}
\rhead{}
\lfoot{}
\cfoot{第\thepage 页 \quad 共 \pageref{LastPage} 页}
\rfoot{}
\renewcommand{\headrulewidth}{0pt}
\renewcommand{\headwidth}{\textwidth}
\renewcommand{\footrulewidth}{0pt}

\usepackage{framed} 

% 中文数字的section 序号
\usepackage{zhnumber} 
\renewcommand{\thesection}{\zhnum{section} }
\renewcommand{\thesubsection}{\arabic{section}.\arabic{subsection}}

\renewcommand{\normalsize}{\fontsize{12}{18}\selectfont}


%\title{深圳大学考试答题纸}
%\author{(以论文、报告等形式考核专用)}

\begin{document}

\begin{center}
    \LARGE{深圳大学考试答题纸}\normalsize

    (以论文、报告等形式考核专用)

    二〇二X$\thicksim$二〇二X学年度第一学期
\end{center}

\fontsize{10.5}{16}

\noindent\kaishu{课程编号\underline{(课程编号)} 课序号\underline{01} 课程名称\underline{(课程名称)} 主讲教师\underline{(教师)} 评分\underline{   }} % 评分中的 是全角空格,所以会在pdf中显示为空格,之后的同理

\noindent\kaishu{学  号\underline{ (学号) } 姓名\underline{ (姓名) } 专业年级\underline{ (专业年级) }}

\begin{framed}
    \noindent\kaishu{教师评语:}
    \quad\\[1cm]
\end{framed}
%\end{comment}

\begin{center}
    \noindent\kaishu{题目:}

    \heiti{\fontsize{14}{21}标题标题标题标题}
\end{center}

\normalsize
\normalfont


正文


%\newpage
%\bibliographystyle{unsrt}
%\bibliography{./reference.bib}

\end{document}