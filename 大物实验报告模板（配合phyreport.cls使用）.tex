\documentclass[a4paper,zihao=5,UTF8,fontset=fandol]{phyreport}
% 修改为 ../phyreport 如果你像推荐流程那样cls文件在上级目录

%\usetikzlibrary{decorations.pathreplacing}
\usetikzlibrary{decorations.pathmorphing}
% 实验预习处: http://172.31.80.14:7101
% 快乐制表网站: https://www.tablesgenerator.com/#

\expName{(这里修改为实验名称)}
\expAuthor{(这里修改为名字)}
\expStuID{(学号)}
\expNum{(组号)}
\expTchr{(这里修改为教师名字)}
\expDate{}{}{} %大概位置是不对的因为我习惯手写
\subDate{}{}{}
\expID{(课程编号)}
\expType{(实验类型)}

\begin{document}

\phyExpCover

% 插入预习报告
% \includeFullPageGraphicsBeta{预习报告文件路径}

\fancypage{\fbox}{} % 边框

\section{实验设计方案}

\subsection{实验目的}

\subsection{实验原理}

\subsection{实验仪器}

\subsubsection{实验装置}

\subsubsection{选用仪器列表}

\smartLongLine
\section{实验内容及具体步骤}

\smartLongLine
\section{数据记录及数据处理}

\smartLongLine
\section{实验结果陈述与总结}

\subsection{结果陈述}

\subsection{实验总结}

\endBox

% 如果需要下面这段原始数据记录表请取消注释

%\newpage
%\thispagestyle{empty}

%\fancypage{ }{}
%\section*{原始数据记录表}
%https://www.tablesgenerator.com/#

%\noindent 组号 \rule[-5pt]{2cm}{0.4pt} \ \ 姓名 \rule[-5pt]{2cm}{0.4pt} \ \ 实验名称 \rule[-5pt]{5cm}{0.4pt}



\end{document}
